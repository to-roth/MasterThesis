\chapter{Overall Conclusion}
\label{global:conlusion}
\section{Summary and conclusion}
This thesis contributes research in the field of graph neural networks and quantum computing as well as the prediction of molecular properties. First, classical graph-based deep learning methods from the existing literature are analyzed and prototypically implemented. Later, these findings will be combined with quantum computing principles to investigate a problem from computational chemistry and materials science. The focus is on hydrogen compounds and their potential energy surfaces, which are an indicator for catalytic processes in hydrogen production. \\ 

This cumulative thesis is divided into two papers, each describing the development and evaluation of models for the given problem. In the first paper, a classical graph-based convolutional neural network model is implemented and tested using the QM9 dataset and a simpler water dataset, each describing a collection of molecules with quantum chemical properties. This model serves as a fundamental step in demonstrating the effectiveness of GNN architectures in capturing the relationships within molecular structures and enabling the prediction of molecular properties. The second paper provides insights into the field of quantum computing and attempts to extend the convolutional GNN architecture to its quantum counterpart, a Quantum Graph Neural Network, using the limited literature available.
In order to summarize the findings of this thesis and its contribution to the field of research, the following artifacts were created during the research process:

\begin{itemize}
    \item literature review of different graph neural network architectures
    \item prototypical implementation of a graph neural network for the potential energy prediction
    \item baseline neural networks for the comparison to graph-based models with different datasets
    \item foundation paper analysis of different papers in the context of quantum graph neural networks
    \item based on the foundation paper analysis: requirements analysis to identify the requirements for the development of a quantum graph neural network that is able to predict molecular properties 
    \item attempt to develop a quantum graph neural network based on the conducted requirement analysis
    \item implementation of a quantum convolutional neural network architecture as a basis for further development on quantum graph-based models
\end{itemize} 

Although the realization of a QGNN could not be fully implemented due to various challenges, such as the limited literature on the integration of quantum computing and graph-based machine learning, a prototype quantum convolutional neural network architecture was successfully developed. This prototype represents an important step towards utilizing the potential of quantum computing for graph-based information processing at the molecular and atomic level. As shown, several artifacts were created as part of this work. Thus, this work contributes to the field of hydrogen research and the prediction of molecular properties by combining the theoretical investigation of the problem at hand with the practical development and evaluation of models not only on classical computers but also in quantum simulation. \\

In summary, this work not only highlights the potential synergies between quantum computers and neural networks for comprehensive molecular predictions, but also provides a foundation and path for future research. Overall, the development of complex deep learning algorithms to utilize quantum hardware is still at an early stage. Further advances in quantum computing and quantum machine learning will be necessary to fully exploit the potential of e.g. quantum graph-based neural networks in computational chemistry and thus harness the power of quantum computing and achieve the quantum advantage status in the future. 


%\newpage
%\section{Critical Reflection}
%\newpage
%\section{Outlook}
%\newpage

